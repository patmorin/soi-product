\documentclass{patmorin}
\listfiles
\usepackage{pat}
\usepackage{paralist}
\usepackage{dsfont}  % for \mathds{A}
\usepackage[utf8x]{inputenc}
\usepackage{skull}
\usepackage{paralist}
\usepackage{graphicx}
\usepackage[noend]{algorithmic}

\usepackage[normalem]{ulem}
\usepackage{cancel}
%\usepackage{enumitem}

\usepackage{todonotes}

% etoolbox allows for robust commands that don't need \protect, e.g.
% \newrobustcmd{\onesub}{\mathord{\includegraphics{figs/one-sub}}}
% \subsection{Approximate Voronoi Diagrams in $G^{\onesub}_k$}
\usepackage{etoolbox}

\usepackage[longnamesfirst,numbers,sort&compress]{natbib}

\usepackage[mathlines]{lineno}
\setlength{\linenumbersep}{2em}
% \linenumbers
% \rightlinenumbers
% \linenumbers
\newcommand*\patchAmsMathEnvironmentForLineno[1]{%
 \expandafter\let\csname old#1\expandafter\endcsname\csname #1\endcsname
 \expandafter\let\csname oldend#1\expandafter\endcsname\csname end#1\endcsname
 \renewenvironment{#1}%
    {\linenomath\csname old#1\endcsname}%
    {\csname oldend#1\endcsname\endlinenomath}}%
\newcommand*\patchBothAmsMathEnvironmentsForLineno[1]{%
 \patchAmsMathEnvironmentForLineno{#1}%
 \patchAmsMathEnvironmentForLineno{#1*}}%
\AtBeginDocument{%
\patchBothAmsMathEnvironmentsForLineno{equation}%
\patchBothAmsMathEnvironmentsForLineno{align}%
\patchBothAmsMathEnvironmentsForLineno{flalign}%
\patchBothAmsMathEnvironmentsForLineno{alignat}%
\patchBothAmsMathEnvironmentsForLineno{gather}%
\patchBothAmsMathEnvironmentsForLineno{multline}%
}


% Taken from
% https://tex.stackexchange.com/questions/42726/align-but-show-one-equation-number-at-the-end
\newcommand\numberthis{\addtocounter{equation}{1}\tag{\theequation}}


\setlength{\parskip}{1ex}

% Document-specific commands and math operators
\DeclareMathOperator{\tw}{tw}
\DeclareMathOperator{\td}{td}
\DeclareMathOperator{\chicen}{\chi_{\mathrm{cen}}}
\DeclareMathOperator{\chilin}{\chi_{\mathrm{lin}}}
\DeclareMathOperator{\dist}{dist}
\DeclareMathOperator{\vor}{Vor}
\DeclareMathOperator{\interior}{int}

\newrobustcmd{\onesub}{\mathord{\includegraphics{figs/one-sub}}}
\newrobustcmd{\leftup}{\mathord{\includegraphics{figs/left-up}}}

\title{\MakeUppercase{The Structure of Planar Sphere-of-Influence Graphs}\thanks{This research was partly funded by NSERC.}}
\author{Prosenjit Bose, Vida Dujmović, Pat Morin, and David R. Wood}

\date{}


\begin{document}

\maketitle

\begin{abstract}
    The sphere-of-influence graph of points in $\R^2$ has a product structure theorem?
\end{abstract}

\section{Introduction}

The \emph{intersection graph} of a set $S$ of sets is the graph $G$ with vertex set $V(G):=S$ and edge set $\{vw\in\binom{S}{2}: v\cap w\neq\emptyset\}$.  In this paper we consider the class $\mathcal{G}$ of intersection graphs of sets $D$ of open disks in $\R^2$ with the restriction that no disk in $D$ contains the center of any other disk in $D$.

The graph class $\mathcal{G}$ is closely related to the sphere-of-influence graphs of points in $\R^2$, defined as follows:  Let $P\subseteq\R^d$ be a finite point set.  The \emph{sphere-of-influence graph} $G$ of $P$ is the graph with vertex set $V(G):=P$ that contains an edge $vw$ if and only if there exists open balls $d_v$ and $d_w$ centered at $v$ and $w$, respectively, such that $d_v\cap d_w\neq \emptyset$, $d_v\cap P=\{v\}$ and $d_w\cap P=\{w\}$.  Sphere-of-influence graphs are the edge-maximal graphs in $\mathcal{G}$, in the following sense: For any $G\in\mathcal{G}$, the sphere-of-influence graph of the centers of the disks in $V(G)$ contains $G$ as a subgraph.  The edge-density of sphere-of-influence graphs in Euclidean and non-Euclidean settings has received considerable attention  \cite{avis.horton:,michael.quint:sphere,soss:on,ismailescu.kim.ea:improved, dwyer:expected}.
% More references here: https://www.semanticscholar.org/paper/Sphere-of-influence-graphs%3A-Edge-density-and-clique-Michael-Quint/308a798c83230a30428b89da04121be072f95705


\section{The Product Structure of Sphere-of-Influence Graphs}

Let $P:=\{p_1,\ldots,p_n\}$ be a set of points in $\R^2$ and $D:=\{d_1,\ldots,d_n\}$ be a set of open disks such that, for $i\in\{1,\ldots,n\}$,  $d_i$ is the maximum-radius disk centred at $p_i$ with $d_i\cap P=\{p_i\}$, and let $c_i$ the circle that is the boundary of $d_i$.  Let $G$ be the sphere-of-influence graph of $P$.  (Equivalently, $G$ is the intersection graph of $\{d_1,\ldots,d_n\}$.)  We want to prove that there exist a graph $H$ of treewidth at most $?$ and a path $P$ such that $G$ is isomorphic to a subgraph of $H\boxtimes P$. 

With a slight abuse of notation, we will treat the vertices of $G$ as the integers $1,\ldots,n$, so that, if $v=p_i$, then $p_v$ and $d_v$ denote $p_i$ and $d_i$ respectively.  We may assume that $G$ is connected since we can always add a path points to $p$ to make a it so.  In particular, for any two connected components $C_1$ and $C_2$ of $\bigcup_{i=1}^n d_i$, one can always add a set $X$ of points so that the sphere of influence graph of $P\cup X$ is a supergraph of $G$ in which $V(C_1)$ and $V(C_2)$ are in the same connected component.

So that we have a place to start, we let $t>0$ and assume that $p_1:=(0,t\sqrt{7/4})$, $p_2:=(-t. -t\sqrt{3/2})$, and $p_3:=(t, -t\sqrt{3/2})$.  For a sufficiently large value of $t$, the points $p_1,p_2,p_3$ are the vertices of an equilateral triangle whose interior contains the disks $d_4\ldots,d_n$ and $p_1p_2p_3$ is a $3$-cycle in $G$.  This assumption is safe, since we can always add three such vertices to $P$ to obtain a graph that contains $G$ as a subgraph.  Let $L_0,\ldots,L_h$ be a BFS layering of $G$ with $L_0:=\{p_1,p_2,p_3\}$ and let $T$ be a spanning forest of $G$ in which each vertex $v\in L_i$ has a parent in $L_{i-1}$, for each $i\in\{1,\ldots,h\}$.

Let $c$ be a simple closed curve in $\R^2$, and let $V_c:=\{v\in V(G):\overline{d_v}\cap c\neq\emptyset\}$.  A partition $(V_1,V_2,V_3)$ of $V_c$ is \emph{Sperner-valid} for a bounded component $X$ of $\R^2\setminus (c\cup \bigcup_{v\in V_c} d_v)$ if the boundary $\partial X$ of $X$ can be partitioned into three simple curves $f_1$, $f_2$, and $f_3$ such that $f_i\subseteq c\cup \bigcup_{v\in V_i}c_v$.  A partition $(V_1,V_2,V_3)$ of $V_c$ is \emph{Sperner-valid} if it is Sperner-valid for each bounded component of $\R^2\setminus (c\cup \bigcup_{v\in V_c} d_v)$.

Let $c_0$ be the circle that contains $p_1$, $p_2$, and $p_3$.   Observe that $D_{c_0}:=\{d_1,d_2,d_3\}$ and that $(\{d_1\},\{d_2\},\{d_3\})$ is a Sperner-valid partition of $D_{c_0}$.

We say that a subset $S\subseteq V(G)$ has \emph{thickness} $k$ with respect to $T$ if there exists $k$ disjoint vertical paths $P_1,\ldots,P_k$ in $T$ such that each, for each $x\in S$, $d_x \cap (\bigcup_{i=1}^k\bigcup_{vw\in P_i}\bigcup_{i=1}^k vw)\neq\emptyset$.  That is, each disk in $D$ defined by a center in $S$ intersects at least one of the edges of $P_1,\ldots,P_k$.

\begin{lem}
    Let $c\subseteq c_0\cup\interior(c_0)$ be a simple closed curve in $\R^2$, let $(D_1,D_2,D_3)$ be a Sperner-valid partition of $D_c$, and let $D':=\{d\in D:d\cap (\overline{c}\cup c^-)\neq\emptyset\}$.  Then the induced graph $G':=G[D']$ has a vertex partition $\mathcal{P}$ such that 
    \begin{compactenum}[(i)]
      \item $D_1,D_2,D_3 \in \mathcal{P}$;
      \item each part $P\in\mathcal{P}\setminus\{D_1,D_2,D_3\}$ has thickness at most 3; and
      \item the quotient graph $H:=G'/\mathcal{P}$ has a width-$3$ tree decomposition $\mathcal{T}:=(B_x:x\in V(T))$ in which $\{D_1,D_2,D_3\}\subseteq B_x$ for some $x\in V(T)$.
    \end{compactenum}
\end{lem}

\begin{proof}
  Since $c\subseteq c_0\cup\interior(c_0)$, each vertex $v\in V(G')$ has a $T$-ancestor $w$ with $d_w\in D$.  If $w$ is the lowest such $T$-ancestor of $v$ and $w\in D_i$ then we say that $v$ has colour $i$.
  
  
  
  For each vertex $v\in V(G')$, consider the lowest ancestor $w$ of $v$ in $T$ that is in $D'$.  Since $c\subseteq c_0\cup\interior(c_0)$, $w$ is well defined.  We say that $v$ has colour $i$ if $w\in D_i$.  
  
  For each point 
  
  
  
  For each point $p\in\R^2$ and any disk $d$ of radius $r$ with center $c$, define $\dist(p,d):=\|pc\|_2-r$.  
  
  
    $w\in D'$ on the path from $v$ to the root of $T$.
We now extend this colouring of $c_0'$, first to a colouring of the edges and vertices of $G$ and then to the entire interior of $X$. 
\begin{enumerate}
  \item The vertices $p_1$, $p_2$, and $p_3$ are coloured red, green, and blue, respectively.
  \item For each $i\in\{1,\ldots,h\}$ and each vertex $v\in L_i$, the colour of $v$ is the the same as that of its parent in $T$.
  \item For each edge $uv$ of $T$, the colour of $uv$ is the same as the colour of its two endpoints.
\end{enumerate}

We treat $G$ as a geometric graph whose vertices are points and whose edges are line segments joining their endpoints.  Thus, the colouring defined above defines a colouring of the boundary of $X$ and a one-dimensional subset of the interior of $X$.  Now, for each uncoloured point in the interior of $x$ we find the largest open disk $d_x$ centered at $x$ that contains no coloured point in its interior.  Let $P_x$ be the set of coloured points on the boundary of $d_x$.  By definition $P_x$ is non-empty.  If all points in $P_x$ have the same colour, $\alpha$, then we colour $x$ with $\alpha$. Otherwise, we leave $x$ uncoloured.  

At the end of this process, all but a one-dimensional subset of the interior of $X$ remains uncoloured.  Sperner's Lemma implies that there is an uncoloured point $s$ in the interior of $X$ such that every disk of positive radius centered at $s$ contains points of all three colours.  From $x$ we construct 

\end{proof}



\bibliographystyle{plainurlnat}
\bibliography{soi-product}




\end{document}
