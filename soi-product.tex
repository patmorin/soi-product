\documentclass{patmorin}
\listfiles
\usepackage{pat}
\usepackage{paralist}
\usepackage{dsfont}  % for \mathds{A}
\usepackage[utf8x]{inputenc}
\usepackage{skull}
\usepackage{paralist}
\usepackage{graphicx}
\usepackage[noend]{algorithmic}

\usepackage[normalem]{ulem}
\usepackage{cancel}
%\usepackage{enumitem}

\usepackage{todonotes}

% etoolbox allows for robust commands that don't need \protect, e.g.
% \newrobustcmd{\onesub}{\mathord{\includegraphics{figs/one-sub}}}
% \subsection{Approximate Voronoi Diagrams in $G^{\onesub}_k$}
\usepackage{etoolbox}

\usepackage[longnamesfirst,numbers,sort&compress]{natbib}

\usepackage[mathlines]{lineno}
\setlength{\linenumbersep}{2em}
% \linenumbers
% \rightlinenumbers
% \linenumbers
\newcommand*\patchAmsMathEnvironmentForLineno[1]{%
 \expandafter\let\csname old#1\expandafter\endcsname\csname #1\endcsname
 \expandafter\let\csname oldend#1\expandafter\endcsname\csname end#1\endcsname
 \renewenvironment{#1}%
    {\linenomath\csname old#1\endcsname}%
    {\csname oldend#1\endcsname\endlinenomath}}%
\newcommand*\patchBothAmsMathEnvironmentsForLineno[1]{%
 \patchAmsMathEnvironmentForLineno{#1}%
 \patchAmsMathEnvironmentForLineno{#1*}}%
\AtBeginDocument{%
\patchBothAmsMathEnvironmentsForLineno{equation}%
\patchBothAmsMathEnvironmentsForLineno{align}%
\patchBothAmsMathEnvironmentsForLineno{flalign}%
\patchBothAmsMathEnvironmentsForLineno{alignat}%
\patchBothAmsMathEnvironmentsForLineno{gather}%
\patchBothAmsMathEnvironmentsForLineno{multline}%
}


% Taken from
% https://tex.stackexchange.com/questions/42726/align-but-show-one-equation-number-at-the-end
\newcommand\numberthis{\addtocounter{equation}{1}\tag{\theequation}}


\setlength{\parskip}{1ex}

% Document-specific commands and math operators
\DeclareMathOperator{\tw}{tw}
\DeclareMathOperator{\td}{td}
\DeclareMathOperator{\chicen}{\chi_{\mathrm{cen}}}
\DeclareMathOperator{\chilin}{\chi_{\mathrm{lin}}}
\DeclareMathOperator{\dist}{dist}
\DeclareMathOperator{\vor}{Vor}

\newrobustcmd{\onesub}{\mathord{\includegraphics{figs/one-sub}}}
\newrobustcmd{\leftup}{\mathord{\includegraphics{figs/left-up}}}

\title{\MakeUppercase{The Structure of Planar Sphere-of-Influence Graphs}\thanks{This research was partly funded by NSERC.}}
\author{Prosenjit Bose, Vida Dujmović, Pat Morin, and David R. Wood}

\date{}


\begin{document}

\maketitle

\begin{abstract}
    The sphere-of-influence graph of points in $\R^2$ has a product structure theorem?
\end{abstract}

\section{Introduction}

The \emph{intersection graph} of a set $S$ of sets is the graph $G$ with vertex set $V(G):=S$ and edge set $\{vw\in\binom{S}{2}: v\cap w\neq\emptyset\}$.  In this paper we consider the class $\mathcal{G}$ of intersection graphs of sets $D$ of open disks in $\R^2$ with the restriction that no disk in $D$ contains the center of any other disk in $D$.

The graph class $\mathcal{G}$ is closely related to the sphere-of-influence graphs of points in $\R^2$, defined as follows:  Let $P\subseteq\R^d$ be a finite point set.  The \emph{sphere-of-influence graph} $G$ of $P$ is the graph with vertex set $V(G):=P$ that contains an edge $vw$ if and only if there exists open balls $d_v$ and $d_w$ centered at $v$ and $w$, respectively, such that $d_v\cap d_w\neq \emptyset$, $d_v\cap P=\{v\}$ and $d_w\cap P=\{w\}$.  Sphere-of-influence graphs are the edge-maximal graphs in $\mathcal{G}$, in the following sense: For any $G\in\mathcal{G}$, the sphere-of-influence graph of the centers of the disks in $V(G)$ contains $G$ as a subgraph.  The edge-density of sphere-of-influence graphs in Euclidean and non-Euclidean settings has received considerable attention  \cite{avis.horton:,michael.quint:sphere,soss:on,ismailescu.kim.ea:improved, dwyer:expected}.
% More references here: https://www.semanticscholar.org/paper/Sphere-of-influence-graphs%3A-Edge-density-and-clique-Michael-Quint/308a798c83230a30428b89da04121be072f95705


\section{The Product Structure of Sphere-of-Influence Graphs}

Let $P:=\{p_1,\ldots,p_n\}$ be a set of $n$ points in $\R^2$ and, for each $i\in\{1,\ldots,n\}$, let $d_i$ be the open disk centered at $p_i$ with $d_i\cap P=\{p_i\}$ and having maximal radius.  Let $G$ be the sphere-of-influence graph of $P$.  (Equivalently, $G$ is the intersection graph of $\{d_1,\ldots,d_n\}$.)  We want to prove that there exist a graph $H$ of treewidth at most $?$ and a path $P$ such that $G$ is isomorphic to a subgraph of $H\boxtimes P$. We may assume that $G$ is connected since, otherwise we can handle each connected component of $G$ separately.

So that we have a place to start, we let $t>0$ and assume that $p_1:=(0,t\sqrt{7/4})$, $p_2:=(-t. -t\sqrt{3/2})$, and $p_3:=(t, -t\sqrt{3/2})$.  For a sufficiently large value of $t$, the points $a_0,a_1,a_2$ are the vertices of an equilateral triangle whose interior contains the disks $d_4\ldots,d_n$.  This assumption is safe, since we can always add three such vertices of $P$ to obtain a graph that contains $G$ as a subgraph.

Let $C_0$ be the circle that contains $p_1$, $p_2$, and $p_3$.



\bibliographystyle{plainurlnat}
\bibliography{soi-product}




\end{document}
